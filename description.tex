\documentclass[10pt,a4paper]{article}
%\usepackage[utf8]{inputenc}
\usepackage[english]{babel}
%\usepackage[OT1]{fontenc}
\usepackage{amsmath}
%\usepackage{amsfonts}
\usepackage{amssymb}
\usepackage[left=20mm,right=20mm, top=20mm,bottom=15mm,bindingoffset=0cm]{geometry}


\usepackage[T1]{fontenc}
%\usepackage{lmodern}

\title{Optimal P2P network}
\begin{document}
\maketitle
Suppose we have a network of $N$ computers, and one of them (or several) has a file that we want to distribute to every other peer. The sooner all the computers will have the file the better, and this is the optimization task we want to solve. 

Our base model for this task will be continuous, and later we will discretize the problem. Connections between peers can be modelled via graph: vertices of the graph will represent the computers in our network, and weights of the edges is the bandwidth between the pair of the computers. Let the adjacency matrix of this graph be $B$. Besides the bandwidth of pairs we also have connection speed of every computer to the network. This values will be assigned to vertices. Let $u_i$ be maximum upload speed for the computer $i$, and $d_i$ be the maximum download speed for the computer $i$. 

The last part of our model (and the beginning of discretization) is the file description. All we need to now is it's length, we will name it $L$. We will split the file in $K$ parts and will deliver the parts along the net. Discretizing time and recomputing the bandwidth and all upload and download restrictions in terms of amount of parts (which will be integers), we now ready to formulate the problem.

Let $x_{ikt}$ be a variable that is equals to $1$ when computer $i$ has a part $k$ at the moment of time $t$ and to $0$ otherwise, and $y_{ijkt}$ be a variable that is equals to $1$ when part $k$ was transfered from computer $i$ to computer $j$ at moment of time $t$. In terms of $x$ and $y$ we want to minimize the following thing:
\begin{equation}
	\max_{i} \min \{t: \sum_k x_{ikt} = K \} \to \min_{x, y}.
\end{equation} 

In other words, we want to minimize the time when all computers in network will have all parts of the file. Now let's formulate the restrictions on the variables $x, y$:
\begin{align}
	\label{bandwidth}	\sum_k y_{ijkt} \leqslant & \ b_{ij} \\ 
	\label{download}	\sum_{ik} y_{ijkt} \leqslant & \ d_j \\ 
	\label{upload}	\sum_{jk} y_{ijkt} \leqslant & \ u_i \\ 
	\label{nonerasability}	\forall t' \geqslant t \  x_{ikt'} \geqslant & \ x_{ikt} \\ 
	\label{availability}	y_{ijkt} \leqslant & \ x_{ikt} \\
	\label{binary} x_{ikt}, y_{ijkt} \in & \ \{0, 1 \}.
\end{align}

The description of every inequality is the following:
\begin{itemize}
	\item Equation \eqref{bandwidth} is the bound on the amount of parts that can be transfered within connection between $i$ and $j$.
	\item Equations \eqref{download} and \eqref{upload} restrict the uploads and downloads for computer $i$.
	\item Equations \eqref{nonerasability} forbid the disappearance of the part $k$ from computer $i$.
	\item Equations \eqref{availability} state that part $k$ can be downloaded from comuter $i$ if and only if this part has been downloaded to the computer $i$ before.
	\item Equations \eqref{binary} requires the $x$ and $y$ to be binary.
\end{itemize}

Every restriction is linear in terms of $x$ and $y$. To finalize this problem as a linear programming problem, we need to reduce the objective to a linear form. We use the following expression:
\begin{equation}
	\max_{i} \min \{t: \sum_k x_{ikt} = K \} \to \min_{x,y} \Rightarrow \max_i \sum_t \Big( K - \sum_k x_{ikt} \Big) \to \min_{x, y}
\end{equation}

First of all, it is not the same objective, but it's close: when computer $i$ downloaded file, the expression $K - \sum_k x_{ikt}$ tends to zero, so the sooner the the computer download the file the better, but not in the same way better. Still, it's close.

This is the final problem:
\begin{align*}
	\begin{array}{rrll}
	& \min_{x, y, \xi} \xi \\ 
	\ \\
	& \sum\limits_k y_{ijkt} \leqslant & \mkern-12mu b_{ij} \\ 
	& \sum\limits_{ik} y_{ijkt} \leqslant & \mkern-12mu d_j \\ 
	s. t. & \sum\limits_{jk} y_{ijkt} \leqslant & \mkern-12mu u_i \\ 
	& x_{ikt'} \geqslant & \mkern-12mu x_{ikt} & \mkern-32mu \forall t' \geqslant t  \\ 
	& y_{ijkt} \leqslant & \mkern-12mu x_{ikt} \\
	& x_{ikt}, y_{ijkt} \in & \mkern-12mu \{0, 1 \}.
	\end{array} 
\end{align*}
\end{document}